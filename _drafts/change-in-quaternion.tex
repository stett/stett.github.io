%% LyX 2.0.5.1 created this file.  For more info, see http://www.lyx.org/.
%% Do not edit unless you really know what you are doing.
\documentclass[english]{article}
\usepackage[T1]{fontenc}
\usepackage[latin9]{inputenc}
\usepackage{listings}
\usepackage{geometry}
\geometry{verbose,tmargin=2cm,bmargin=3cm,lmargin=2cm,rmargin=2cm,headheight=2cm,headsep=2cm,footskip=1.5cm}

\makeatletter
%%%%%%%%%%%%%%%%%%%%%%%%%%%%%% User specified LaTeX commands.
\date{}

\makeatother

\usepackage{babel}
\begin{document}
When working with numeric rigid body physics simulations in 2D, it's
as straightforward to update the rotation of a body as it is its linear
position. In 3D, it's not {*}difficult{*} per se, but it can be much
less immediately clear how to go about it, particularly if your basic
transform is stored as a quaternion. This was a sticking point for
me when developing the demo in the video above.

Below I'll derive the change-in matrix, and then the change-in quaternion.
The methods result in similar performance when used with the matching
rotation representation. Note that I'm only developing this mathematically
for explicit, Euler integrators, although it could be pretty simply
adapted to other forms of integration.


\subsection*{Matrix}

If the positions of a vertex in local and world space are $\vec{q}$
and $\vec{r}$, and the center of rotation of the body to which it
belongs is $\vec{c}$, then they are related by

\[
\vec{r}\left(t\right)=R\left(t\right)\vec{q}+\vec{c}\left(t\right)
\]


where $R\left(t\right)$ is the rotation matrix of the body. The linear
velocity of the vertex $\dot{\vec{r}}$ is easy to find, and involves
a term $\dot{R}$, which is the ``change-in'' matrix which we're
looking for. Recall that the linear velocity of a point on a rotating
body is also given by the cross product of the angular velocity $\omega\left(t\right)$
with it's position relative to the center of rotation and making use
of the {[}cross product matrix{]}(https://en.wikipedia.org/wiki/Cross\_product\#Conversion\_to\_matrix\_multiplication),
$\dot{R}$ can be quickly identified.

\begin{eqnarray*}
\dot{\vec{r}}\left(t\right) & = & \dot{R}\left(t\right)\vec{q}+\dot{\vec{c}}\left(t\right)\\
 & = & \omega\left(t\right)\times R\left(t\right)\vec{q}+\dot{\vec{c}}\left(t\right)\\
 & = & \left[\omega\left(t\right)\right]_{\times}R\left(t\right)\vec{q}+\dot{\vec{c}}\left(t\right)
\end{eqnarray*}


Where $\left[\omega\left(t\right)\right]_{\times}$ is the cross product
matrix of $\omega\left(t\right)$. Then $\dot{R}\left(t\right)=\left[\omega\left(t\right)\right]_{\times}R\left(t\right)$,
and the integration step to update the rotation matrix is pretty simple:

\begin{eqnarray*}
R\left(t+\Delta t\right) & = & R\left(t\right)+\Delta t\dot{R}\left(t\right)\\
 & = & R\left(t\right)+\left[\omega\left(t\right)\right]_{\times}R\left(t\right)
\end{eqnarray*}


In a program using {[}GLM{]}(http://glm.g-truc.net/0.9.8/index.html),
this is what the code for the final update step might look like.

\begin{lstlisting}
rotation += glm::matrixCross4(angular_velocity * dt) * rotation;
\end{lstlisting}


Easy peasy!


\subsection*{Quaternion}

We'll develop the quaternion solution in essentially the same way,
taking $Q\left(t\right)$ to be the rotation quaternion of the body.
Remember that to {[}orient a point with a quaternion{]}(https://en.wikipedia.org/wiki/Quaternions\_and\_spatial\_rotation\#Orientation)
the point is reinterpreted as a quaternion with a zero scalar value
and it is multiplied by the rotation on one side and its conjugate
on the other.

\[
\vec{r}\left(t\right)=Q\left(t\right)\vec{q}Q^{-1}\left(t\right)+\vec{c}\left(t\right)
\]


An equivalent operation to the angular velocity cross product which
we used to find the change-in matrix is the {[}angle-axis formula
for quaternions{]}(https://en.wikipedia.org/wiki/Axis\%E2\%80\%93angle\_representation\#Unit\_quaternions).
Then the change-in quaternion can be found by similar identification.

\begin{eqnarray*}
\dot{\vec{r}}\left(t\right) & = & \dot{Q}\left(t\right)\left(Q\left(t\right)\vec{q}\right)\dot{Q}^{-1}\left(t\right)+\vec{c}\left(t\right)\\
 & = & Q_{\times}\left(\vec{\omega}\Delta t\right)\left(Q\left(t\right)\vec{q}\right)Q_{\times}^{-1}\left(\vec{\omega}\Delta t\right)+\vec{c}\left(t\right)
\end{eqnarray*}


where $Q_{\times}$ represents the angle-axis quaternion, whose scalar
and vector parts are given by

\[
Q_{\times}\left(\vec{\omega}\Delta t\right)=\left[\cos\left(\frac{\left|\vec{\omega}\right|}{2}\Delta t\right),\frac{\vec{\omega}}{\left|\vec{\omega}\right|}\sin\left(\frac{\left|\vec{\omega}\right|}{2}\Delta t\right)\right]
\]


Then the quaternion rotation update step would be

\[
Q\left(t+\Delta t\right)=Q\left(t\right)+Q_{\times}\left(\vec{\omega}\Delta t\right)Q\left(t\right)
\]


This is what the code for the final rotation update using quaternions
might look like.

\begin{lstlisting}
float angle = glm::length(angular_velocity);
glm::vec3 axis = angular_velocity / angle;
glm::quat axis_angle(glm::cos(angle * 0.5f), axis * glm::sin(angle * 0.5f));
rotation = normalize(rotation + axis_angle * rotation);
\end{lstlisting}

\end{document}
